% IEEE standard conference template; to be used with:
%   spconf.sty  - LaTeX style file, and
%   IEEEbib.bst - IEEE bibliography style file.
% --------------------------------------------------------------------------

\documentclass[letterpaper]{article}
\usepackage{spconf,amsmath,amssymb,graphicx,listings}

% Example definitions.
% --------------------
% nice symbols for real and complex numbers
\newcommand{\R}[0]{\mathbb{R}}
\newcommand{\C}[0]{\mathbb{C}}

% bold paragraph titles
\newcommand{\mypar}[1]{{\bf #1.}}

% Title.
% ------
\title{JIT-Compiler for Dynamically Controlled Particle Systems}
%
% Single address.
% ---------------
\name{Benjamin Fl\"uck, Jakob Progsch, Simon Laube\thanks{The authors thank Alen Stojanov and Daniele Spampinato for their input during this project}}
\address{Department of Computer Science\\ ETH Z\"urich\\Z\"urich, Switzerland}

% For example:
% ------------
%\address{School\\
%		 Department\\
%		 Address}
%
% Two addresses (uncomment and modify for two-address case).
% ----------------------------------------------------------
%\twoauthors
%  {A. Author-one, B. Author-two\sthanks{Thanks to XYZ agency for funding.}}
%		 {School A-B\\
%		 Department A-B\\
%		 Address A-B}
%  {C. Author-three, D. Author-four\sthanks{The fourth author performed the work
%		 while at ...}}
%		 {School C-D\\
%		 Department C-D\\
%		 Address C-D}
%

\begin{document}
%\ninept
%
\maketitle
%

\begin{abstract}
We explore the application of a custom Just-In-Time (JIT) compiler to
programmable particle systems and compare the resulting performance with
a conventionally optimized C implementation. We show that building a JIT
compiler for our sufficiently narrow scope can match and even exceed the
performance of the C implementation, while lessening the optimization
burden on the programmer.

%~ Describe in concise words what you do, why you do it (not necessarily
%~ in this order), and the main result.  The abstract has to be
%~ self-contained and readable for a person in the general area. You
%~ should write the abstract last.
\end{abstract}

\section{Introduction}\label{sec:intro}

Traditionally environmental animations for games and movies were done procedurally or by hand. 
With the ongoing desire to produce more realistic and detailed environments, these animations are becoming too numerous and complex.
Thus they become too costly to be done procedurally or by hand. 
The steady increase in available compute power in the last few years made it possible to replace these time and cost intensive task with small simulations that produce physically convincing results while still providing a degree of influence to the creator.

\mypar{Motivation}
In the light of these requirements and circumstances we decided to look at rule based particle simulations.
A given set of rules is applied to the particles in each time-step.
These simulations are especially interesting as they allow to model a wide range of environments depending on the set of rules governing a given scenario. 
As the number and composition of such rules is arbitrarily large and can dynamically change over the course of the simulation, it is hard to produce optimized code for such a simulation. We have therefore decided to explore the possibility of creating a Just-In-Time compiler (JIT) specialized for such particle simulations.

Limiting the scope of the compiler input to a well defined and limited problem allows our JIT compiler to incorporate a lot of domain specific knowledge and optimizations directly into its inner workings. For example, our JIT compiler knows the data layout for the particles and does not have to support any number of possible container or data formats.

\mypar{Related work}
JIT compilers are located somewhere between traditional Ahead-Of-Time compilers (AOT) such as clang\cite{clang}, gcc\cite{gcc} or icc\cite{icc} and interpreters for scripting languages such as JavaScript. As such they often include techniques from both end of this spectrum. The core idea is that code is only translated to machine instructions at the latest possible time. This makes JIT compilers well suited for dynamic languages such as JavaScript or Lua as well as situations in which information about the target architecture is either not available beforehand or can be used to increase performance. JIT compilers are widely used, most notably in the form of the Java Virtual Machine\cite{jvm} and the JavaScript\cite{v8}\cite{SpiderMonkey} engines found in web browsers. Another state of the art JIT compiler from which we took some inspiration for the internal code represenation is LuaJIT \cite{LuaJIT}\cite{LuaJITir}, a JIT compiler for the Lua language. Other uses of JIT compilers that resemble our use case are found in graphics card drivers to translate generic shader code (OpenGL) and compute kernels (OpenCL) to highly optimized hardware specific machine code.

%~ As for particle systems they have been studied in great details and been in use for a long time $>-$citation needed$<-$ and there exist optimizations and implementations for any number of simulation settings.


\section{Idea and Background}\label{sec:background}

In this section we provide a detailed description of the particle system we use, the domain specific language (DSL) the JIT compiler accepts as input, how the JIT compiler is structured, and conduct a cost and complexity analysis.

\mypar{Particle System}
The particle simulation we concentrate on consists of two main components, the particles themselves and the set of rules governing the interactions and progress of the simulation. The particles are provided in the following way:

\noindent\begin{minipage}{\linewidth}
\begin{lstlisting}
struct particle_t {
    float position[3];
    float mass;
    float velocity[3];
    float charge;
} particle;
\end{lstlisting}
\end{minipage}

This allows for 16 bytes aligned access to both the position and velocity vectors, as well as creating rules for diverse application, e.g. for physical or chemical simulations.

As for the rules, they are provided as functions that take a particle and an array of rule specific values as input.
This function then computes and applies the rule specific update to the given particle. This setup allows for a great flexibility on what the rules can do, and even the time step for the simulation itself is such a rule:

\noindent\begin{minipage}{\linewidth}
\begin{lstlisting}
newton_step_apply(particle *p, void *d){
    float dt =  d[0];
    p->position[0] += dt*p->velocity[0];
    p->position[1] += dt*p->velocity[1];
    p->position[2] += dt*p->velocity[2];
}
\end{lstlisting}
\end{minipage}

In the end the complete simulation follows three basic steps:
\begin{lstlisting}
   (1) get particles
   (2) get rules
   (3) iterate over all particles 
       and apply every rule to them
\end{lstlisting}

If the environment changes between iterations one has to update the list of rules to adapt the simulation to the new circumstances.
The simulation can then proceed immediately afterwards.

\mypar{DSL}
To conveniently define the rules for the JIT compiler we introduced a domain specific language (DSL). The DSL provides a subset of C operator syntax including arithmetic, bitwise and ternary operators, temporary float variables and array access on the input arrays. This feature set is sufficiently expressive for the intended purposes while still being easy and fast to parse.

\mypar{JIT Compiler}
In contrast to the static optimized code, the JIT compiler exploits possible optimisations across multiple rules by fusing all the rules into one executable function. %The goal of our JIT compiler is to fuse multiple rules into one executable function, therefore exploiting possible optimizations across multiple rules which static optimized code cannot do in ahead of time compilations.
To translate the DSL into their intermediate representation, a straight forward hand written single pass lexer and recursive descent parser was written. %Due to the limited scope of the DSL a straight forward hand written single pass lexer and recursive descent parser is sufficient to translate the rules into their intermediate representation.

For the intermediate representation a static single assignment form representation (SSA-IR)\cite[Chapter~6.2.4]{dragon}\cite{LuaJITir} is used since it allows efficient implementation of the relevant optimization passes. Because the DSL does not contain flow control constructs, the optimizer and code generator only ever deal with a single basic block. This significantly reduces the analysis that is required to optimize the code as compared to a general purpose compiler. The last step of translating the SSA-IR to machine code is performed using a x86-64 instruction encoder (PLASM). The code generator assigns registers and stack spilling locations on the fly and emits VEX encoded instructions only. The VEX encoding allows the use of non destructive instructions which are very close to their SSA-IR counterparts. To make the code actually executable it is wrapped in a function template and written to memory, which is then switched to executable state by means of a system call.

%~ MORE HERE
%~ -> standard parser/lexing
%~ -> processing/optimizations (dead code/ deduplication)
%~ -> Outputting code
%~ -> setting exec bit

\mypar{Cost Analysis}
While it is possible to determine the exact op-count for any given rule, the final op-count depends on the number and nature of the involved rules and can change throughout the simulation.
Therefore there is no global cost measure.
The op-count relative to the input size, in our case number of particles, is always $O(1)$.
For all except the smallest simulations including only very few rules, the neglected constant factor pushes our computation far into the compute bound region.
Furthermore our simulation exhibits perfect spatial locality as it passes over the particles linearly.
Given that our JIT compiler changes the number of stores, loads, and operations required for applying the rules, it is not possible to simply compare the performance in flops/cycle between our jited code and static optimized code.
The reason simply being that code with a lower performance but requiring fewer operations may still be faster than code requiring more operations and achieving higher operational intensity.

We have therefore decided to look at the runtime, more precisely at the cycles per particle. This allows for a precise comparison how our jited code performs in comparison to the conventional code in an absolute way.


\section{Implementation Details}\label{sec:method}

Starting from a simple, straight-forward baseline implementation we have introduced two main paths of optimization which we discuss in detail in this section. On one hand we applied conventional optimizations i.e. static optimizations that are applied at compile time, to the baseline implementation. On the other hand we have the JIT compiler itself, which implements all the necessary steps to dynamically produce executable code.

\mypar{Baseline Implementation}
Our baseline implementation consists of all the possible rules, each accessible through a function pointer, and an array of particles. In order to run the simulation one can now simply collect a list of desired rules and pass them to the main simulation function. This function then simply iterates over all particles and applies every rule to each particle.

\mypar{Conventional Optimizations}
The dynamic nature of the rule composition limits a conventional compiler or the programmer to optimizing the code on a per rule basis. With the baseline code already written in SSA form (single static assignment) we have then implemented two variants of vectorization. The first one implementing 3D-vector arithmetic in SSE code and the second one processing multiple particles in parallel.

Implementing 3D-vector arithmetic may seem attractive but poses two problems. SSE registers can hold four floats (single precision floating point number) in 128 bits which means for 3D-vectors we already waste 1/4th of the register space and going to the newer AVX instruction at 256 bit wide registers means either wasting 5/8th of the available space or rewriting all the code to process two 3D-vectors in parallel. Furthermore there are no dedicated instructions to calculate the euclidean length of a vector and therefore require significant overhead for data shuffling. The one advantage of this approach lies in the treatment of conditional code, e.g. in collision detections. As only one particle is processed at a time it is possible to skip unneeded computations.

Processing multiple particles by packing their respective x, y, and z coordinates into different register reverses these drawbacks and advantages. This treatment allows to go from SSE to AVX instructions by loading twice as many particles into the registers. In contrast it is no longer possible to use conditional statements to skip portions of the code. A conditional check may be true for some particles and false for others. Instead we use the conditional check to produce a mask indicating for which particles the check holds and for which it doesn't. We then run the conditional code for all particles and mask the application of the results according to this indication mask.

\mypar{JIT Compiler}
The JIT compiler exclusively uses the multiple particle approach for vectorization which allows the intermediate (SSA-IR) code to only deal with the scalar form of the rules that are then executed on 1, 4 or 8 particles at a time. This means vectorization exclusively happens in final code generation and is of no concern for the optimization passes implemented on the SSA-IR form.

The rules are initially given as strings in a strongly reduced C style syntax. For example the euler update rule looks as follows:

\noindent\begin{minipage}{\linewidth}
\begin{lstlisting}
dt = [8]
[0] = [0] + dt*[4]
[1] = [1] + dt*[5]
[2] = [2] + dt*[6]
\end{lstlisting}
\end{minipage}

The numbers in brackets are indexing into implicit arrays that have their meaning assigned by the code generator. In our case indices 0-7 refer to particle components and higher indices refer to the rule specific input values.

These rules are then translated to SSA-IR form by a single pass recursive descent parser. The SSA-IR form is stored as an array of fixed width 64bit instructions which contain an opcode, flags and up to three operands. The operands are indices that either refer to previous instructions in the SSA-IR array or to the input arrays in case of the load and store instructions. Using this flat and fixed width representation allows efficient implementation of optimizations passes.

\mypar{Optimization}
The first optimization pass consists of removing redundant stores and loads. Since concatenating rules will result in multiple redundant loads (almost every rule will load position for example) and stores (only the last store to any component will actually have an effect on the memory), a significant amount of those operations can be combined or removed. Similarly redundant computations which show up as identical instructions in the SSA-IR can also be skipped. Both of these operations are done by updating a index remapping array. All uses of redundant instrunctions are remapped to use the first instance of that instruction. As a result the additional instances are not referred to anymore and leaves them as dead code.\\
In the next step instructions are recursively marked as being live starting from the store instructions. The live instruction can then be compacted again removing all the dead code including the redundant operations from the previous steps. Depending on the rule combination this process can remove up to half the instructions of the original rules. However, it does not reduce the length of the critical path.\\
The last step before code generation is the scheduler. The scheduler tries to heuristically reorder the instructions into a better sequence. Since the processors we are targeting have reasonably large reorder buffers (168 slots for Intel Ivy Bridge microarchitecture) the exact ordering between single instructions is not that important. Still we want largely independent paths of execution to be interleaved in the greater scheme since the loop bodies as a whole easily exceed the reorder buffers size. Additionally the live ranges of values can be changed by rescheduling, which allows the scheduler to influence the register pressure the code generator will have to deal with.
The scheduler therefore operates by enumerating all eligible instructions that have their dependencies fulfilled and assigns a score to them. It then schedules the instruction with the highest score and repeats the process until all instructions are scheduled. \\
There are two factors that influence the score: criticality and influence on live registers. Criticality is the sum of latencies of dependent instructions. Influence on live registers is the change in the amount of live registers the instruction results in when scheduled. The scoring therefore prefers instructions that have high criticality and reduce the amount of live registers.

\mypar{Code Generation}
Each of the three vectorization levels (scalar, 4-wide, 8-wide) have their own code generator. The main difference between the versions is how they translate load and store instructions since the non-scalar versions need to gather the particle components from memory into registers. Instruction selection is done by a lookup table since the SSA-IR instructions have already been chosen to directly correspond to the underlying x86-64 instruction set. The main challenge therefore is the allocation and spilling of registers. To utilize the instruction sets ability to directly use memory operands for arithmetic instructions, load operations are deferred to the first usage of their result. When spilling is necessary, the register that has its next use the farthest from the current position is assigned a free slot on the stack and moved there. In addition to the translation of the block itself, the code generator also generates some fixed code such as the loop structure and function boilerplate around the optimized rules, to make them a callable function following the system-V AMD64 ABI. This way the resulting buffer can simply be cast to an appropriate function type and called from the C code. The code generator allocates memory using the \texttt{mmap} system call to obtain page aligned memory with read and write permissions. Before returning, that memory is changed to read and execute permissions using the \texttt{mprotect} system call. This is necessary since memory obtained from \texttt{mmap} will typically not have execute permissions and calling into it would therefore result in segmentation faults.
%~ For this class, explain all the optimizations you performed. This mean, you first very briefly
%~ explain the baseline implementation, then go through locality and other optimizations, and finally SSE (every project will be slightly different of course). Show or mention relevant analysis or assumptions. A few examples: 1) Profiling may lead you to optimize one part first; 2) bandwidth plus data transfer analysis may show that it is memory bound; 3) it may be too hard to implement the algorithm in full generality: make assumptions and state them (e.g., we assume $n$ is divisible by 4; or, we consider only one type of input image); 4) explain how certain data accesses have poor locality. Generally, any type of analysis adds value to your work.
%~
%~ As important as the final results is to show that you took a structured, organized approach to the optimization and that you explain why you did what you did.
%~
%~ Mention and cite any external resources including library or other code.
%~
%~ Good visuals or even brief code snippets to illustrate what you did are good. Pasting large amounts of code to fill the space is not good.


\section{Experimental Results}\label{sec:exp}

Here you evaluate your work using experiments. You start again with a
very short summary of the section. The typical structure follows.

\mypar{Experimental setup} Specify the platform (processor, frequency, cache sizes)
as well as the compiler, version, and flags used. I strongly recommend that you play with optimization flags and consider also icc for additional potential speedup.

Then explain what input you used and what range of sizes. The idea is to give enough information so the experiments are reproducible by somebody else on his or her code.

\mypar{Results}
Next divide the experiments into classes, one paragraph for each. In the simplest case you have one plot that has the size on the x-axis and the performance on the y-axis. The plot will contain several lines, one for each relevant code version. Discuss the plot and extract the overall performance gain from baseline to best code. Also state the percentage of peak performance for the best code. Note that the peak may change depending on the situation. For example, if you only do additions it would be 12 Gflop/s
on one core with 3 Ghz and SSE and single precision floating point.

Do not put two performance lines into the same plot if the operations count changed significantly (that's apples and oranges). In that case first perform the optimizations that reduce op count and report the runtime gain in a plot. Then continue to optimize the best version and show performance plots.

{\bf You should}
\begin{itemize}
\item Follow the guide to benchmarking presented in class, in particular
\item very readable, attractive plots (do 1 column, not 2 column plots
for this class), proper readable font size. An example is below (of course you can have a different style),
\item every plot answers a question, which you pose and extract the
answer from the plot in its discussion
\end{itemize}
Every plot should be discussed (what does it show, which statements do
you extract).

\section{Conclusions}

In this paper we have introduced the idea of using a just in time compiler for creating simulations of dynamic particle systems. We have then outlined how such a purpose built JIT compiler can take advantage of implicit knowledge and simplified programming requirements. Furthermore we have shown that even a simple implementation can already produce code that is performing at least equivalently or better than conventionally optimized code. Besides direct advantages in performance having such a JIT compiler also relieves the programmer from having to optimize every new simulation rule he might come up with. After the initial effort to create such a JIT compiler its usage provides the best bang for buck for the programmer, he gets the performance of heavily optimized code at the cost of a simple and straight forward implementation. Originally both the conventional code and the jit compiler were producing SSE code using 128bit wide register with the difference that the JIT compiler only needed little work and changes in the backend to produce new AVX code while it would be necessary to go over all existing code and redo all the work for the conventional code. This further highlights the advantages of building such an infrastructure not only to gain immediate performance gains but also savings for future adjustments when they come along. 

As it is clear that it is not possible to produce such a JIT compiler with all the possible bells and whistles that exist for compilers we have nevertheless shown the value and possible gains for such a endeavour. In a larger context this also furthers the view that automatic code generation and optimization 

\mypar{Future Work} We have already highlighted the possibility of further loop unrolling which allowed the conventional code to partially close the gap to our jit compiled code. In general there is a huge amount of work and papers around concerning compilers which provides for a treasure trove of possible enhancements for any new and straight forward compiler such as ours. 

%~ \section{Further comments}
%~
%~ Here we provide some further tips.
%~
%~ \mypar{Further general guidelines}
%~
%~ \begin{itemize}
%~ \item For short papers, to save space, I use paragraph titles instead of
%~ subsections, as shown in the introduction.
%~
%~ \item It is generally a good idea to break sections into such smaller
%~ units for readability and since it helps you to (visually) structure the story.
%~
%~ \item The above section titles should be adapted to more precisely
%~ reflect what you do.
%~
%~ \item Each section should be started with a very
%~ short summary of what the reader can expect in this section. Nothing
%~ more awkward as when the story starts and one does not know what the
%~ direction is or the goal.
%~
%~ \item Make sure you define every acronym you use, no matter how
%~ convinced you are the reader knows it.
%~
%~ \item Always spell-check before you submit (to me in this case).
%~
%~ \item Be picky. When writing a paper you should always strive for very
%~ high quality. Many people may read it and the quality makes a big difference.
%~ In this class, the quality is part of the grade.
%~
%~ \item Books helping you to write better: \cite{Higham:98} and \cite{Strunk:00}.
%~
%~ \item Conversion to pdf (latex users only):
%~
%~ dvips -o conference.ps -t letter -Ppdf -G0 conference.dvi
%~
%~ and then
%~
%~ ps2pdf conference.ps
%~ \end{itemize}
%~
%~ \mypar{Graphics} For plots that are not images {\em never} generate (even as intermediate step)
%~ jpeg, gif, bmp, tif. Use eps, which means encapsulate postscript, os pdf. This way it is
%~ scalable since it is a vector graphic description of your graph. E.g.,
%~ from Matlab, you can export to eps or pdf.

%Here is an example of how to get a plot into latex
%(Fig.~\ref{fftperf}). Note that the text should not be any smaller than shown.

%\begin{figure}\centering
%  \includegraphics[scale=0.33]{dft-performance.eps}
%  \caption{Performance of four single precision implementations of the
%  discrete Fourier transform. The operations count is roughly the
%  same. {\em The labels in this plot are too small.}\label{fftperf}}
%\end{figure}



% References should be produced using the bibtex program from suitable
% BiBTeX files (here: bibl_conf). The IEEEbib.bst bibliography
% style file from IEEE produces unsorted bibliography list.
% -------------------------------------------------------------------------
\bibliographystyle{IEEEbib}
\bibliography{bibl_conf}

\end{document}

