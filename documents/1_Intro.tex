\section{Introduction}\label{sec:intro}

Traditionally environmental animations for games and movies were done procedurally or by hand. With the ongoing desire to produce realistic and detailed environments these animations are becoming to numerous and complex to be done by hand or procedurally, not to mention too costly as well. The steady increase in available compute power in the last few years made it possible to replace these time and cost intensive task will small simulations that produce physically convincing results while still providing a degree of influence to the creator. 

\mypar{Motivation} 
In the light of these requirements and circumstances we decided to look at particle simulations. Such particle simulations are especially interesting as they allow to model a wide range of environments depending on the set of rules governing a given scenario. As the number and composition of such rules is arbitrarily large and can dynamically change over the course of the simulation it is not possible to produce optimized code for such a simulation. We have therefore decided to explore the possibility of creating a just-in-time compiler (JIT) specialized for such particle simulations.

Limiting the scope of the compiler input to a well defined and small problem allows our JIT compiler to incorporate a lot of domain specific knowledge and optimizations directly into its inner workings. For example our JIT compiler knows the data layout for the particles and does not have to support any number of possible containers or data formats. 

\mypar{Related work} 
JIT compilers are located somewhere between traditional ahead-of-time compilers such as gcc or icc and interpretors for scripting languages such as javascript. And as such often include techniques from both end of this spectrum. One of the best known and best performing JIT compiler is luajit $->$citation needed: luajit$<-$, a JIT compiler for the scripting language lua.

As for particle systems they have been studied in great details and been in use for a long time $>-$citation needed$<-$ and there exist optimizations and implementations for any number of simulation settings.